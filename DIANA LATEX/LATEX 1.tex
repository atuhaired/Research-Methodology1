\documentclass[11pt,A4paper]{article}
\usepackage{zed-csp,graphicx,color}%from
\begin{document}
\begin{titlepage}
  \begin{figure}[h]
  \centerline{\small MAKERERE 
  \includegraphics[width=0.1\textwidth]{muk_log} UNIVERSITY}
\end{figure}
\centerline{COLLEGE OF COMPUTING AND INFORMATIC SCIENCES}
\paragraph{•}
\centerline{DEPARTMENT OF COMPUTER SCIENCE\\}
\paragraph{•}

\centerline{COURSEWORK: RESEARCH METHODOLOGY(BIT 2207)\\}
\paragraph{•}

\centerline{LECTURER: MR.ERNEST MWEBAZE}
\paragraph{•}

\centerline{TOPIC\\}ASSESSING THE PERCEPTION OF INFORMATION COMPUTER TECHNOLOGY BETWEEN GENDER IN KAMPALA \\
\paragraph{•}
\centerline{COMPILED BY: \
 ATUHAIRE DIANA}
 \paragraph{•}
\centerline{STUDENT NUMBER : 216002722}
\paragraph{•}
\centerline{REGISTRATION NUMBER:16/U/3826/eve}
\paragraph{•}
\end{titlepage}
\pagenumbering{roman}
\tableofcontents
\newpage
\pagenumbering{arabic}
\section{TOPIC}
 Assessing the perception of Information Computer Technology between gender in Kampala
\section{PROBLEM STATEMENT}
ICT has extensively been done by males at the expense of females in Kampala District. This has led to deterioration of computer technological knowhow among females that is: 60% of females of the Ugandan population were computer ignorant (National Information Technology Authority Report –NITA, 201 4) thus reducing different opportunities that would arise especially in the Education and Business Sector in Kampala District.
\section{MAIN OBJECTIVE}
To assess the perceptive of ICT between gender in Kampala District
\section{SPECIFIC OBJECTIVES}
 1. To establish or determine the role of Information Computer Technology
 2. To determine the level at which Information Computer Technology is being used between genders in Kampala District. 
3. To establish the effort of ICT between gender in Kampala District

\section{JUSTIFICATION}
ICT involves concepts, methods and applications consistently evolving on an almost daily basis. The broadness of ICT covers any product that will store, retrieve manipulate, transmit or receive information electronically in a digital form. For example, personal TV, Digital TV, emails, robots etc. therefore ICT should be taken up no matter gender since it is an enhanced mode of communication being used worldwide. Since there are no studies having been carried out to assess the perception of ICT between genders; this research aims at establishing the roles, effects and level at which ICT is perceived in Kampala District. 
\section{SCOPE}
This study was limited to the Western region of Uganda. Emphasis will be drawn on determining the role and effect of ICT between gender in Kampala District.
\section{METHODOLOGY}
\subsection{Research Design}
A research design refers to systematic plan drawn by the researcher during the research study (Garwood and Jupp 2011). Generally the kind of data used in this study was both quantitative and qualitative .Qualitative research design was employed to get experience view points and suggestion towards the role of Information Technology.
\subsection{Population Size}
This research wasconducted on different services provided with the use of information technology in Kampala 
\subsection{Sampling Frame}
The sampling frame was limited to Kampala City running projects since Information Technology development starts within this area. 
\subsection{Research Procedure}
\subsection{Desk Study}
This study mainly considered reports from Ministry of Information and Technology

\subsection{Data collection methods}
Both primary and secondary data was used to obtain information for purposes of study. Secondary dataincludedinformation from text books, highway manuals, journals and reports whereas primary data included moderate field surveys

\subsection{Data Processing and Analysis}

All collected information from the survey was recorded, checked and verified for the analysis. Results were then presented using pie charts, graphs and tables to interpret variations and relationships between the genders







\end{document}